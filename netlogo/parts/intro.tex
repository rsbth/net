\section{Introduction}

\subsection{Context}
Signing a paper contract between two users may be very simple, but signing a paper contract between 2 000 signers is very difficult. The same problematic arise if the contract involves only 10 people scattered around the globe. In such cases the ability to sign a contract using a computer becomes very handy. One thing necessary for such a protocol is \textit{fairness}, that is signer A can not get the signature of any other honest signer unless signer A has committed to the contract. An easy way to solve this problem would be for every signer to send his signature to a \textit{Trusted Third Party} (TTP), then the TTP sends back a fully signed contract to everyone. However, by doing this the TTP would become an necessary element of any contract signature and would end up being a bottleneck. Another approach is to share the load between all signers, and to refer to the TTP only in case of problem during the signature. This kind of MPCS is called \textit{optimistic}.

\subsection{Background}
We are interested in the notion of \textit{abuse-freeness} as defined in \cite{Promise}. A protocol is abuse-free if no group of signer can prove that he holds the power to complete or abort the contract signature. Garay et al. introduce a new cryptographic object called \textit{Private Contract Signature} in \cite{Promise} based on ElGamal crypto system \cite{Elgamal} and use it to define a two-party contract signing and a three-party contract signing. These construction are proven to be fair, optimistic and abuse-free. In \cite{asokan} an optimistic protocol for exchanging fairly signatures was proposed by Asokan et al..  Mukhamedov et al. proposed another optimistic MPCS for any number of signers in \cite{muka} also base on private contract signatures. Wang proposed an abuse-free, optimistic two-party contract signing in \cite{wang-rsa} using RSA and trapdoor commitment schemes \cite{tcs}. Kordy et al. proved an equivalence between a mathematical sequence and the fairness of an MPCS protocol in \cite{Original}, it is based on private contract signature. The obtained protocol is abuse-free, optimistic, fair and efficient because it reaches the lower bounds in terms of bandwidth and message complexity determined by Garay et al. in \cite{bounds}. In \cite{DAG-MPCS} Mauw et al. extend the work of Kordy et al. using a labeled DAG instead of a linear sequence, and achieving as well an optimistic abuse-free fair and efficient MPCS.

\subsection{Objectives}
A common point to optimistic MPCS is that at some point, a commitment is exchanged before sending the signature. At first, \textit{verifiable escrows} \cite{asokan} and \textit{verifiable encrypted signature} \cite{ves} were used as commitments but the consequent schemes were not abuse-free.
RSA crypto system is now an industry standard, but there are only two protocols which use it and guarantee abuse-freeness and optimism, however it only works for a two-party contract as show in table \ref{table:results}.
This paper focuses on finding an MPCS protocol which is abuse-free, optimistic and fair for any number of signers. A secondary objective is its efficiency, \textit{i.e.} it has to be usable in practice without heavy computation and should stick to the RSA industry standard.

\subsection{Methods}
We chose contract signing as our research field because it is a very specific field and it will probably become more and more crucial with time. It involves heavy cryptography and mathematics which makes it very interesting to work in.
We started by looking at the first articles we could find about contract signing on IEEE Xplore in order to gain some basic knowledge about it. From there, we were able to orient our research to be more and more specific, by adding the "optimistic" and searching on Scopus. There were 63 results, so we looked at the older ones first. This allowed us to add the "abuse-free" keyword and get down to 13 results (14 but one is plagiarised from another). After applying our inclusion and exclusion criteria, we were left with 8 articles. By doing a forward search from \cite{Promise}, we were able to find two more articles that are expected to be abuse-free free, though the proof wasn't in the paper.

\subsection{Results}
A summary of the included papers in this SLR is provided in table \ref{table:results}. We discuss the method for obtaining those in section IV. and V. We discuss the results in section VII. and give limits of this paper in section VIII.
