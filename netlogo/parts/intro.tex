\section{Introduction}

Dining philosophers problem is a synchronisation problem, popular in the computer science domain. The problem starts with philosophers meeting for spaghetti on a round table, who can be only in one of the three states, namely thinking, hungry and eating. Each philosopher in the problem is provided with a fork next to him and in order to eat the spaghetti, the philosopher will require two forks. It is obvious from the problem that all philosophers cannot access (eat) the spaghetti at the same time. The forks in the problem can be visualised as a shared resource. Generally, there are two issues with the setting namely starvation and deadlock. Starvation can be viewed as one of the philosophers being in the hungry state for a relatively large interval with the neighbour not releasing the fork. 
\\Deadlock is a situation when philosophers hold a fork and doesn’t release it until they get the other, this state is called circular wait and leads to a deadlock. Although resources are available, it is possible that neither of the philosophers get a chance to access the spaghetti due to deadlock. This in computer science can be viewed in terms of shared resources and processes trying to access the shared resource at the same time. Deadlocks are not desirable due to the fact that idle time for the shared resource will be high, resulting in  a high response time. Therefore, we need a strategy from [1] to to make sure that at least one hungry philosopher can always eat and also that all philosophers will get the eat almost same amount of spaghetti on average. There is also one more feature in the system that will help to find the solution, i.e when a philosopher is holding a fork, he can place a mark on it or remove the existing mark. The marks are assumed to be clearly visible to other philosophers who see the fork. It is also assumed that one fork is always randomly marked for a philosopher but the marked forks cannot be visually distinguished in case there is no cooperation. We have chosen this model as it is for us to visualise the model using a modelling tool -Net logo- and understand the needs and gaps for designing a high performance schedular. Net logo is a integrated modelling environment  that can be used to program agents with the help of an agent based programming language. The advantage with net logo is the interactive (User Interface) UI.         


